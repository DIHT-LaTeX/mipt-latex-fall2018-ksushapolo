\documentclass[12pt]{article}

\usepackage[russian]{babel}

\title{Домашнее задание №1}
\author{Половченя Ксения}
\date{}

\begin{document}
	\maketitle
	\begin{flushright}
		{\itshape Audi multa,\\ loquere pauca}
	\end{flushright}
	\vspace{20pt}
	Это мой первый документ в системе компьютерной вёрстки \LaTeX.
	
	\begin{center}
		{\LARGE \sffamily <<Ура!!!>>}
	\end{center}
	\vspace{15pt}
	
	А теперь формулы. {\scshape Формула}~--- краткое и точное словесное выражение, определение или же ряд математических величин, выраженный условными знаками.
	\vspace{15pt}
	
	\hspace{28pt}{\Large \bfseries Термодинамика}
	
	Уравнение Менделеева--Клайперона~--- уравнение состояния идеального газа, имеющее вид $p V = \nu R T$, где $p$~--- давление,  $V$~--- объём, занимаемый газом, $T$~--- температура газа, а $R$~--- универсальная газовая постоянная.
	\vspace{15pt}
	
	\hspace{28pt}{\Large \bfseries Геометрия \hfill Планиметрия}
	
	Для {\slshape плоского} треугольника со сторонами $a$, $b$ и $c$ и углом $\alpha$, лежащим против стороны $a$, справедливо соотношение 
	$$a^2 = b^2 + c^2 - 2 b c \cos\alpha,$$ из которого можно выразить косинус угла треугольника: $$\cos\alpha = \frac{b^2 + c^2 - a^2}{2 b c}.$$
	Пусть $p$~--- полупериметр треугольника, тогда путём преобразований можно получить, что $$\tg\frac{\alpha}{2} = \sqrt{\frac{(p - b)(p - c)}{p (p - a)}},$$
	
	\vspace{1cm}
	\begin{flushleft}
		На сегодня, пожалуй, хватит \ldots Удачи!
	\end{flushleft}
	
\end{document}